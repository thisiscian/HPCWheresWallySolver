
Region Detection

This algorithm is intended to find connected regions of pixels.
The input is a single channel binary matrix; each pixel has a single value of either 0 or 255.
A region is defined as a group of pixels, each with a value of 255, that are connected through nearest neighbours.

Regions are detected by multiplying each pixel of the input with a unique integer value, starting at $p_{00}=1$.
Thus $p_{ij} = (Ni+j+1)*p_{ij}$, where $N$ is the range of values that $j$ can take.
This means that each pixel now either has a value of 0, or it's own unique value.
From this point, each non-zero pixel of the matrix is looped over, and it's value is changed to the maximum of it's four neighbours.
This loop is repeated until no pixels are changed.
The algorithm takes the maximum value of neighbouring pixels, and ignores non-zero pixels, so maximal values can not be spread outside of a region's boundary.
Within a region, it is evident that every pixel will have the value of the maximum pixel within that region.
As each pixel has a unique value, it follows that each regional maximum will also have a unique value.
By counting the unique values in the matrix, the number of regions can be found.
Similarly, properties of the region can be calculated by analysing specific unique values.
For example, the mean position of the pixels within the region, the size of the region, and the bounding box around the region.

This algorithm is very useful for augmenting the results of colour detection.
Colour detection schemes generally output a binary matrix of results, based on whether each pixel passes the criteria.
Detecting regions can help to make the results more understandable to the user, and also condense the results into a smaller format.
