\documentclass[../main.tex]{subfiles}
\begin{document}
  \section{Implementation}
    \subsection{Red And White}
      Regions with red and white stripes can be found using a few simple techniques.
      Here, we find two masks (arrays with binary values) that describe the location of sufficiently red and white regions in the image.
      The term sufficiently is used, as defining 'red' and 'white' is not a simple task for a computer.
      The function \texttt{get\_colour\_in\_image}, discussed in section \ref{function_getcolinimg}, covers the issues with colour definition.

      Once the masks have been created, they are vertically blurred using a Gaussian blur.
      This is done so that when the two masks are multiplied together, there will be non-zero values found in overlapping areas.
      The multiplication now describes regions that have red and white regions of colour that are above each other.
      As seen in figure \ref{rw_stripe_ex}, the region found may not encompass the entire area desired.
      This is solved by blurring the located stripe, making a combined region with nearby stripes.
      The regions are then identified using the \texttt{find\_region\_from\_mask} function.
    \subfile{implementation/colour_implementation}
    \subfile{implementation/red_and_white}
    \subfile{implementation/blue_trousers}
    \subfile{implementation/find_glasses}
    \subfile{implementation/find_features}

\biblio
\end{document}
