\documentclass[../main.tex]{subfiles}
\begin{document}
  \section{Introduction}
    Computer vision is a field concerned with using computers to process and analyse images. 
    It is an an attempt to augment human abilities to perceive visual information,
      allowing users to increase their capabilities to accurately and quickly use visual techniques. %% sentence too long

    Recognising an object is one of the most important things that human vision is able to do.
    This means that the brain is able to decipher the basic information received from the eyes, deducing any objects the information describes.
    This is an intricate task and not simply replicated programmatically.

    %% Reliably finding an object within an image is difficult to do at speed.
    For each way to describe an object, several different techniques emerge to locate the object.
    Each technique has varying requirements and reliability.
    Some need specific descriptions (e.g. a sample image) and produce very reliable results.
    Others do not require such precise descriptions, but do not create similarly dependable results.
    By combining these techniques together, the reliability and robustness of computational object recognition can be improved.
    Using all available techniques will considerably increase the time the vision program takes to complete.
    This can be mitigated by computing each techniques in parallel.
    An intuitive method of parallelism is the task farm.
    This means that general tasks (in this case, each method of identifying an object) will be run concurrently.
    Task farms are useful in this case, because it allows the use of serial algorithms libraries that would otherwise complicate or prevent parallelism.
    Computing results simultaneously, a task farm allows computer vision programs to produce results both quickly and reliably.

    %% Where's Wally is a good basis for parallel object recognition
    %%    - maybe add a bit about clearly definable features?
    A good way of testing the task farm method is with Where's Wally puzzles.
    These are simple cartoons, normally a large image filled with various characters, who wear simply coloured clothing, see Figure \ref{justachump}.
    One of these characters is the eponymous Wally, who is dressed distinctly from most others characters, see Figure \ref{justwally}.
    Similarly dressed characters exist, Figure \ref{justwenda}, adding some complexity to finding Wally correctly.
    The cartoon nature of the characters means that shapes are boldly coloured and often bordered by a black line.
    As Where's Wally is a puzzle, sometimes Wally will be hard to find; he is often obscured, camouflaged or simply small.
    Creating a program to solve Where's Wally is non-trivial, requiring a combination of computer vision techinuqes.
    Despite this, the puzzle provides a simple base to implement parallel object recognition.

    \begin{figure}[H]
    \centering
      \begin{subfigure}[b]{0.3\textwidth}
        \centering
        \includegraphics[height=0.15\textheight]{just_a_chump}   
        \caption{A normal person}
        \label{justachump}
      \end{subfigure}
      \begin{subfigure}[b]{0.3\textwidth}
        \centering
        \includegraphics[height=0.15\textheight]{just_wally}   
        \caption{Wally}
        \label{justwally}
      \end{subfigure}
      \begin{subfigure}[b]{0.3\textwidth}
        \centering
        \includegraphics[height=0.15\textheight]{just_wenda}   
        \caption{Wenda}
        \label{justwenda}
      \end{subfigure}
    \caption{Characters from Where's Wally}
    \label{wallychars}
    \end{figure}

    %% Real World Uses of object recognition/parallel object recognition
    Extending beyond Where's Wally puzzles, parallel object recognition is a useful technique for many areas.
    Wildlife conservation is a delicate task.
    To accurately know which species are endangered, it is important to have an accurate count of the members of the species for a given region.
    Actively surveying the population, by means of physically interacting with members can have adverse effects on the population.
    Nielsen \cite{electrofishing} discusses the negative effects of electrofishing on rare fish populations.
    She goes on to indicate the lack of non-invasive methods of surveying the population, without which population counts cannot be maintained.
    Directly surveying endangered species can be inefficient, slow or dangerous to either the researcher or the animal in question.
    Passive techniques, such as photography, allow the researcher to estimate populations without interacting with the environment.
    Ideally a researcher would be constantly vigilant and able be to immediately identify each species correctly.
    This is rarely the case; a single human is fallible and a team may be beyond the funding of the endeavour.
    Instead, with access to any modern laptop and a digital camera, parallel computer vision may be able to assist in many ways.
    A video feed would allow observation for as long as the battery lasts, and a database of the features of regional species would help with identification.
    Parallel species recognition would allow multiple species to be surveyed at a time.
    It could even allow non-experts to survey the populations, freeing up researchers for more critical tasks.
      
    TRAFFIC JAMS
    
    FINDING CHILDREN LOST IN A CROWD


\end{document}
